\documentclass[a4paper,10pt]{article}
\usepackage[T1]{fontenc}
\usepackage[utf8]{inputenc}
\usepackage[polish]{babel}

\begin{document}

  \section{Fundamentalne prawo dynamiki płynu nieściśliwego}

	\begin{equation}
		\rho=const
	\end{equation}

	\begin{equation}
		\nabla\cdot\vec{u} = 0
	\end{equation}

	\begin{equation}
		\frac{\partial\vec{u}}{\partial t} = -\frac{1}{\rho}\nabla p + \nu\nabla^2\vec{u}-\left(\vec{u}\cdot\nabla\right)\vec{u} + \vec{g}
	\end{equation}

  \section{Fundamentalne prawo dynamiki płynu nieściśliwego i nielepkiego}
	\label{sec:fpdpnin}

	\begin{equation}
		\rho=const
	\end{equation}

	\begin{equation}
		\nu=0
	\end{equation}

	\begin{equation}
		\label{eq:fpdpnin:div1}
		\nabla\cdot\vec{u} = 0
	\end{equation}

	\begin{equation}
		\label{eq:fpdpnin:dyn1}
		\frac{\partial\vec{u}}{\partial t} = -\frac{1}{\rho}\nabla p -\left(\vec{u}\cdot\nabla\right)\vec{u} + \vec{g}
	\end{equation}

  \section{Równania N-S w postaci~(\ref{eq:fpdpnin:div1})~i~(\ref{eq:fpdpnin:dyn1}) dla przestrzeni 2D}

	\begin{equation}
		\label{eq:rns2du1}
		\frac{\partial u}{\partial t} = -\frac{\partial u^2}{\partial x} - \frac{\partial uv}{\partial y} - \frac{\partial \phi}{\partial x} + g_x 
	\end{equation}

	\begin{equation}
		\label{eq:rns2dv1}
		\frac{\partial v}{\partial t} = -\frac{\partial uv}{\partial x} - \frac{\partial v^2}{\partial y} - \frac{\partial \phi}{\partial y} + g_y 
	\end{equation}

	\begin{equation}
		\frac{\partial u}{\partial x} + \frac{\partial v}{\partial y} = 0
	\end{equation}

  \subsection{Równanie do wyznaczania rozkładu ciśnienia}

	Różniczkując równanie~(\ref{eq:rns2du1}) po zmiennej $x$ otrzymujemy~(\ref{eq:rdwrcu1})

	\begin{equation}
		\label{eq:rdwrcu1}
		\frac{\partial}{\partial t} \frac{\partial u}{\partial x} + \frac{\partial}{\partial x} \left (  \frac{\partial u^2}{\partial x} + \frac{\partial uv}{\partial y} \right )
		+ \frac{1}{\rho} \frac{\partial^2 p}{\partial x^2} = 0
	\end{equation}

	Różniczkując równanie~(\ref{eq:rns2dv1}) po zmiennej $y$ otrzymujemy~(\ref{eq:rdwrcv1})

	\begin{equation}
		\label{eq:rdwrcv1}
		\frac{\partial}{\partial t} \frac{\partial v}{\partial y} + \frac{\partial}{\partial y} \left (  \frac{\partial uv}{\partial x} + \frac{\partial v^2}{\partial y} \right )
		+ \frac{1}{\rho} \frac{\partial^2 p}{\partial y^2} = 0
	\end{equation}

	Dodając stronami~(\ref{eq:rdwrcu1})~i~(\ref{eq:rdwrcv1})

	\begin{equation}
		\label{eq:rdwrc1}
		\frac{\partial}{\partial t} \left ( \frac{\partial u}{\partial x} + \frac{\partial v}{\partial y} \right ) 
		+ \frac{\partial}{\partial x} \left (  \frac{\partial u^2}{\partial x} + \frac{\partial uv}{\partial y} \right )
		+ \frac{\partial}{\partial y} \left (  \frac{\partial uv}{\partial x} + \frac{\partial v^2}{\partial y} \right )
		+ \frac{1}{\rho} \left (  \frac{\partial^2 p}{\partial x^2} + \frac{\partial^2 p}{\partial y^2} \right ) = 0
	\end{equation}

	Oznaczając

	\begin{equation}
		D = \frac{\partial u}{\partial x} + \frac{\partial v}{\partial y}
	\end{equation}

	oraz

	\begin{equation}
		Q = \frac{\partial}{\partial x} \left (  \frac{\partial u^2}{\partial x} + \frac{\partial uv}{\partial y} \right )
		+ \frac{\partial}{\partial y} \left (  \frac{\partial uv}{\partial x} + \frac{\partial v^2}{\partial y} \right )
		= \frac{\partial^2 u^2}{\partial x^2} + \frac{\partial^2 v^2}{\partial y^2} + 2\frac{\partial^2 uv}{\partial x\partial y}
	\end{equation}

	Otrzymujemy równanie na rozkład cisnienia postaci~(\ref{eq:rdwrc2})

	\begin{equation}
		\label{eq:rdwrc2}
		\frac{\partial}{\partial t}D = -Q - \frac{1}{\rho} \left ( \frac{\partial^2 p}{\partial x^2} + \frac{\partial^2 p}{\partial y^2} \right )
	\end{equation}
	
\end{document}
